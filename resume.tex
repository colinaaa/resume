%!TEX program = xelatex

\documentclass{uniquecv}

\usepackage{fontawesome}
\usepackage{fontspec}
\usepackage[colorlinks,urlcolor=blue]{hyperref}
% ----------------------------------------------------------------------------- %

\setCJKmainfont{SourceHanSerifCN-Regular}
\setmonofont{Consolas} 
\setsansfont{SourceHanSansCN-Regular}
\begin{document}

\name{王清雨}
\medskip

\basicinfo{
  \faPhone ~ (+86) 136-5113-2812
  \textperiodcentered\
  \faEnvelope ~ qingyu.wang@aliyun.com
  \textperiodcentered\
  \faGithub ~ github.com/colinaaa
}

% ----------------------------------------------------------------------------- %

\section{教育背景}
\dateditem{\textbf{清华大学附属中学} \quad 初中、高中}{2012年 -- 2018年 }
成绩:年级前10\%
\dateditem{\textbf{华中科技大学} \quad 计算机科学与技术\quad 本科}{2018年 -- 至今 }
成绩:年级前10\% \quad 英语:暂无


% ----------------------------------------------------------------------------- %

\section{专业技能}
\smallskip
Golang、C/C++、Python、Linux、计算机网络、数据结构、算法

% ----------------------------------------------------------------------------- %

%\section{获奖情况}
%\datedaward{一等奖}{XYZ应用开发大赛} {2026年06月}
%\datedaward{\small{这是一个比较长长长长的奖}}{比较长的奖的比赛}{2016年04月}
%\datedaward{冠军}{武汉KK编程挑战赛} {2015年09月}
%\medskip

% ----------------------------------------------------------------------------- %

\section{项目经历}

% ---
\datedproject{Hackday-Dashboard后台}{团队项目}{2019年05月 -- 至今}
\textit{Go语言后端开发}
\href{https://github.com/colinaaa/UniqueHackDayDashboard-backend}{项目地址}
\vspace{0.4ex}

\begin{itemize}
  \item 使用gin框架
  \item 使用gorm框架
  \item 采用可配置方式启动服务
  \item 使用docker并将镜像压缩在20M以内
  \item 使用docker-compose管理network及volume
  \item 使用阿里云镜像触发器与rancher\ webhook实现自动部署
  \item 完成nginx相关跨域配置
  \item 期间使用mysqldump进行数据迁移
  \item 对项目进行持续运维
\end{itemize}

% ---
\datedproject{吃点啥——智能推荐食品项目}{团队项目}{2019年01月 -- 2019年03月}
\textit{Go语言后端开发}
\href{https://github.com/colinaaa/hackweek}{项目地址}
\vspace{0.4ex}

使用Go语言为ios端提供应用后台,使用docker进行构建及部署。

\begin{itemize}
  \item 采用iris框架
  \item 使用docker及docker-compose进行构建与部署
  \item 数据库使用mongodb进行简单的存储
\end{itemize}

% ---
\datedproject{神奇海螺——问答社区项目}{团队项目}{2018年12月 -- 2018年12月}
\textit{python后台开发}
\href{https://github.com/colinaaa/tasks_uniquestudio}{项目地址}

使用python快速开发,为ios端应用提供后台。
\vspace{0.4ex}
\begin{itemize}
		\item 采用flask框架
		\item 使用jwt对手机端鉴权
		\item 数据库使用sqlite,ORM框架使用SQLAlchemy
\end{itemize}
% ----------------------------------------------------------------------------- %

\section{课外}
\dateditem{\textbf{华中科技大学联创团队}}{2018年10月 -- 至今}
\dateditem{\textbf{计算机学院美团俱乐部}}{2018年10月 -- 至今}

\end{document}

