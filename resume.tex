%!TEX program = xelatex

\documentclass{uniquecv}

\usepackage{fontawesome}
\usepackage{fontspec}
% ----------------------------------------------------------------------------- %

\setCJKmainfont{SourceHanSerifCN-Regular}
\begin{document}

\name{王清雨}
\medskip

\basicinfo{
  \faPhone ~ (+86) 136-5113-2812
  \textperiodcentered\
  \faEnvelope ~ qingyu.wang@aliyun.com
  \textperiodcentered\
  \faGithub ~ github.com/colinaaa
}

% ----------------------------------------------------------------------------- %

\section{教育背景}
\dateditem{\textbf{华中科技大学} \quad 计算机科学与技术\quad 本科}{2018年 -- 至今 }
成绩:年级前10\% %\quad 英语:暂无


% ----------------------------------------------------------------------------- %

\section{专业技能}
\smallskip
C/C++、Python、Golang、Linux


% ----------------------------------------------------------------------------- %

%\section{获奖情况}
%\datedaward{一等奖}{XYZ应用开发大赛} {2026年06月}
%\datedaward{\small{这是一个比较长长长长的奖}}{比较长的奖的比赛}{2016年04月}
%\datedaward{冠军}{武汉KK编程挑战赛} {2015年09月}
%\medskip

% ----------------------------------------------------------------------------- %

\section{项目经历}


% ---
\datedproject{吃点——智能推荐食品项目}{团队项目}{2019年01月 -- 2019年03月}
\textit{Go语言后端开发}
\vspace{0.4ex}

使用Go语言为ios端提供应用后台,使用docker进行构建及部署。

\begin{itemize}
  \item 采用iris框架
  \item 使用docker及docker-compose进行构建与部署
  \item 数据库使用mongodb进行简单的存储
\end{itemize}

% ---
\datedproject{神奇海螺——问答社区项目}{团队项目}{2018年12月 -- 2018年12月}
\textit{python后台开发}

使用python快速开发,为ios端应用提供后台。
\vspace{0.4ex}
\begin{itemize}
		\item 采用flask框架
		\item 使用jwt对手机端鉴权
		\item 数据库使用sqlite,ORM框架使用SQLAlchemy
\end{itemize}


% ---
%\datedproject{Hackday-Dashboard后台}{团队项目}{2019年04月 -- 2019年4月}
%\textit{Go语言后端开发}
%\vspace{0.4ex}

%使用Go语言为团队Hackday的管理平台
%\begin{itemize}
%  \item 支持C11标准大部分标准
%  \item 实现了类似lex/flex的词法解析器生成工具
%  \item 实现了类似yacc/bison的语法解析器生成工具
%  \item 后端部分采用窥孔优化
%\end{itemize}


% ----------------------------------------------------------------------------- %

\section{课外}
\dateditem{\textbf{华中科技大学联创团队}}{2018年10月 -- 至今}

\end{document}

