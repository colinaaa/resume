%!TEX program = xelatex

\documentclass{uniquecv}

\usepackage{fontawesome5}
\usepackage{fontspec}
\usepackage[colorlinks]{hyperref}
% ----------------------------------------------------------------------------- %

%\setmonofont{FiraCode} 
\begin{document}

\name{王清雨}
\medskip

\basicinfo{
  \faPhone ~ (+86) 136-5113-2812
  \textperiodcentered\
  \faEnvelope ~ qingyu.wang@aliyun.com
  \textperiodcentered\
  \faGithub ~ github.com/colinaaa
  \textperiodcentered\
  \faBlog ~ \href{https://outsiders.top}{\color{black}{outsiders.top}}
}

% ----------------------------------------------------------------------------- %

\section{教育背景}
\dateditem{\textbf{清华大学附属中学} \quad 初中、高中}{2012年 -- 2018年 }
成绩:年级前10\%
\dateditem{\textbf{华中科技大学} \quad 计算机科学与技术\quad 本科}{2018年 -- 至今 }
成绩:年级前10\% \quad 英语:CET4-573分


% ----------------------------------------------------------------------------- %

\section{专业技能}
\smallskip
\texttt{Golang、C++1a、Python、Linux、TypeScript、React}、\LaTeX、
计算机网络、操作系统、数据结构、算法

% ----------------------------------------------------------------------------- %

%\section{获奖情况}
%\datedaward{一等奖}{XYZ应用开发大赛} {2026年06月}
%\datedaward{\small{这是一个比较长长长长的奖}}{比较长的奖的比赛}{2016年04月}
%\datedaward{冠军}{武汉KK编程挑战赛} {2015年09月}
%\medskip

% ----------------------------------------------------------------------------- %

\section{个人简介}

我来自联创团队\emph{Web}组,主要专注于后端方向,对前端也有一些简单的涉猎。
\par
能够熟练使用Linux,常用vim进行编程,能够对自己的服务器进行简单的运维。
\par
热爱研究新技术,平常还有研究过\texttt{Haskell, Rust}等技术,能够流利阅读英文文档。
\par
有扎实的计算机理论基础,良好的算法与数据结构基础,了解计算机底层基本原理,喜欢钻研底层原理。


% ----------------------------------------------------------------------------- %


\section{项目经历}

% ---
\datedproject{Caddy-V2}{开源项目}{2019年11月 -- 至今}
\textit{Go语言网络编程}
\quad \href{https://github.com/caddyserver/caddy}{{\color{gray}{\faLink}}~}
\vspace{0.4ex}
\\
Caddy是Golang的一个HTTP Server,我目前主要在为其编写\texttt{TCP/UDP} Proxy模块。
\begin{itemize}
    \item 网络编程
    \item Golang并发编程
\end{itemize}

% ---
\datedproject{Hackday-Dashboard后台}{团队项目}{2019年05月 -- 2019年6月}
\textit{Go语言后端开发}
\quad \href{https://github.com/colinaaa/UniqueHackDayDashboard-backend}{{\color{gray}{\faLink}}~}
\vspace{0.4ex}

\begin{itemize}
  \item 使用gin,gorm框架
  \item 使用阿里云镜像触发器与rancher\ webhook实现自动部署
  \item 对项目进行持续运维
\end{itemize}

% ---
\datedproject{吃点啥——智能推荐食品项目}{团队项目}{2019年01月 -- 2019年03月}
\textit{Go语言后端开发}
\quad \href{https://github.com/colinaaa/hackweek}{{\color{gray}{\faLink}}~}
\vspace{0.4ex}

使用\texttt{Go}语言为ios端提供应用后台,同时使用\texttt{Python}作为胶水代码与AI模型进行交互,采用\texttt{MongoDB}进行数据存储,并使用\texttt{docker}进行构建及部署。

% ---
\datedproject{神奇海螺——问答社区项目}{团队项目}{2018年12月 -- 2018年12月}
\textit{Python后台开发}
\quad \href{https://github.com/colinaaa/tasks_uniquestudio}{{\color{gray}{\faLink}}~}

使用Python快速开发,为ios端应用提供后台。
\vspace{0.4ex}
采用flask框架,使用jwt对手机端鉴权,数据库使用sqlite,ORM框架使用SQLAlchemy
% ----------------------------------------------------------------------------- %

\section{课外}
\dateditem{\textbf{华中科技大学联创团队}}{2018年10月 -- 至今}
\dateditem{\textbf{计算机学院美团俱乐部}}{2018年10月 -- 至今}

\end{document}

