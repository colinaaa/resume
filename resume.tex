%!TEX program = xelatex

\documentclass{uniquecv}

\usepackage{fontawesome5}
\usepackage{fontspec}
\usepackage[colorlinks]{hyperref}
% ----------------------------------------------------------------------------- %

%\setmonofont{FiraCode}
\begin{document}
\name{王清雨}
\medskip

\basicinfo{
  \faPhone ~ (+86) 136-5113-2812
  \textperiodcentered\
  \faEnvelope ~ qingyu.wang@aliyun.com
  \textperiodcentered\
  \faGithub ~ github.com/colinaaa
  \textperiodcentered\
  \faBlog ~ \href{https://about.outsiders.top}{\color{black}{blog}}
}

% ----------------------------------------------------------------------------- %

\section{教育背景}
\dateditem{\textbf{清华大学附属中学} \quad 初中、高中}{2012年 -- 2018年 }
成绩:年级前10\%
\dateditem{\textbf{华中科技大学} \quad 计算机科学与技术\quad 本科}{2018年 -- 至今 }
成绩:年级前10\% \quad 英语:CET4-573分 \quad CET6-551分


% ----------------------------------------------------------------------------- %

\section{专业技能}
\smallskip
\textbf{Programming}
\quad C++1a/Golang/TypeScript/React/
Linux/操作系统/数据结构/算法/计算机网络

\textbf{Tools} \quad \LaTeX/(neo)vim/Git

% ----------------------------------------------------------------------------- %

\section{个人简介}

我来自联创团队\emph{Web}组,主要专注于后端方向,对前端也有一些简单的涉猎。
\par
能够熟练使用Linux,常用vim进行编程,有扎实的计算机理论基础,喜欢钻研底层原理。

% ----------------------------------------------------------------------------- %

\section{实习经历}
\dateditem{\textbf{腾讯科技(北京)有限公司}} {2020年 7月 - 2020年 8 月}

\begin{itemize}
  \item 负责多个Web项目的维护、开发和简单的运维 (DevOps),并进行简单的性能优化(缓存,拆包)
  \item 在两个主要版本内作为主要开发者完成产品迭代,包括前端 \texttt{Vue} 开发与服务端 \texttt{Node.js}开发
  \item 推动组内基础设施方面的建设,包括 CI/CD 等
\end{itemize}

% ----------------------------------------------------------------------------- %

\section{项目经历}

\datedproject{联小创桌游助手}{小程序大赛项目}{2020年10月 - 至今}
{\it 一个桌游发牌小助手}
前端\quad \href{https://github.com/colinaaa/ddwp-frontend}{{\color{gray}{\faLink}}~}
后端\quad \href{https://github.com/colinaaa/ddwp-backend}{{\color{gray}{\faLink}}~}
\quad \emph{前端开发与后端开发}
\begin{itemize}
  \item 前端技术栈:Taro@v2/apollo-client/React(hooks)/WebSocket/TypeScript/Lesscss
  \item 后端技术栈:TypeScript/GraphQL/type-graphql/mongoDB/type-mongoose/Express.js
  \item 将Apollo全家桶移植至小程序平台,包括query/mutation/subscription等绝大多数功能。
  \item 通过小程序提供的API,使用TypeScript实现WebSocket对象,使得同样的代码可以在Web端与小程序端同时运行。
\end{itemize}

\datedproject{Unique HR系统}{团队项目}{2020年2月 - 至今}
{\it 联创团队招新系统前端与后台}
\quad \href{https://github.com/UniqueStudio/UniqueRecruitmentDashboard}{{\color{gray}{\faLink}}~}
\quad \emph{主要维护者}
\vspace{0.4ex}
\\
目前分为三个模块,
{HR Dashboard}\href{https://hr.hustunique.com}{{\color{gray}{\faLink}}~},
{招新表单}\href{https://join.hustunique.com}{{\color{gray}{\faLink}}~}和
{招新后台}\href{https://github.com/UniqueStudio/UniqueRecruitmentBackend}{{\color{gray}{\faLink}}~}。
主要开发工作由学长完成,我负责维护代码,修复bug,前端部署与后端运维,并持续增加新功能

\begin{itemize}
  \item 前端技术栈为 React/TypeScript/Redux/RxJS/Material UI
  \item 后端技术栈为 Node.js/TypeScript/Mongoose/Express.js
  \item 重新实现了招新表单的面试时间选择模块(包括前端和后端),使用类JWT的token来加强安全性
  \item 在HR Dashboard中增加了黑暗模式,同时通过Media Query来自动适配系统黑暗模式
  \item 为招新后台增加发送邮件功能,并使用ACM(Application Config Manager)动态推送配置
\end{itemize}

% ---
\datedproject{UniqueHackday后台}{团队项目}{2019年05月 -- 2019年6月}
\textit{Go语言后端开发}
\quad \href{https://github.com/colinaaa/UniqueHackDayDashboard-backend}{{\color{gray}{\faLink}}~}
\quad \emph{核心开发者与主要运维人员}
\vspace{0.4ex}

UniqueHackday为联创团队举办的hackathon比赛,该项目为比赛提供了报名、组队、数据管理等功能
\begin{itemize}
  \item 开发:使用gin, gorm框架,完成CURD。同时使用casbin进行权限管理
  \item 运维:使用阿里云镜像触发器与rancher\ webhook实现持续交付 (CD)
\end{itemize}

\section{课程实验}
% ---
\datedproject{xv6 Labs}{课内项目}{2020年9月 -- 至今}
{\it MIT 6.S081 Operating System Engineering 课程实验}
\quad \href{https://github.com/colinaaa/xv6-labs-2020}{{\color{gray}{\faLink}}~}
\vspace{0.4ex}
\\
MIT操作系统课程实验,基于UNIX v6 与 RISC-V,全部独立完成。
\begin{itemize}
  \item 熟练掌握xv6系统源代码,RISC-V页表的功能与实现,RISC-V M态,S态,U态的区别。
  \item 了解xv6系统进程调度的方式,了解xv6触发中断,响应中断,中断返回的方式。
  \item page table:实现了一个用户态的page table,使得在内核态不需要进行页表切换即可访问用户态内存。
  \item trap:实现一个自定义的trap,alarm。在n次CPU ticks后,中断用户进程,调用callback函数,然后返回现场。
  \item lazy allocation:实现了类似Linux的lazy allocation,通过page fault来延后分配物理内存的时间。
  \item copy on write:实现了类似Linux的cow,同样通过page fault与page table来提升fork的效率。
\end{itemize}

\datedproject{COOL compiler}{课内项目}{2020年3月 - 2020年4月}
{\it Stanford CS143 Compiler课程实验}
\vspace{0.4ex}
\\
为COOL (Classroom Object Oriented Language)语言实现了一个编译器前端
\begin{itemize}
  \item 使用flex生成词法分析器,了解Context Free Grammar相关概念
  \item 使用bison生成语法分析器,了解Bottom Up parser算法(shift-reduce)
\end{itemize}

\datedproject{TCC}{课内项目}{2020年2月 -- 2020年3月}
{\it C语言编译器前端}
\quad \href{https://github.com/colinaaa/hello-ds}{{\color{gray}{\faLink}}~}
\vspace{0.4ex}
\\
数据结构课程设计,实现了一个C子集的编译器前端。
\begin{itemize}
  \item 编译原理(前端部分),核心代码部分可以自举
  \item 词法分析,语法分析(Top down parser, 递归下降),语义分析(类型判断,常量优化)
\end{itemize}

\datedproject{CSAPP Labs}{课内项目}{2020年1月 -- 2020年3月}
{\it CMU 15213 Introduction to Computer System课程实验}
\vspace{0.4ex}
\\
著名的有关操作系统基础知识的几个实验,全部独立完成。
\begin{itemize}
  \item x86-64汇编的阅读与编写,包括逆向和基础的攻击(栈溢出,Return-oriented programming)
  \item 实现了一个模拟x86-64虚拟内存的虚拟机,并优化代码,以提高其缓存命中率
  \item 实现了支持单线程的{\tt malloc}函数,使用了多个双向循环内核链表,从而兼顾了分配的速率与碎片产生率
  \item 实现了Linux下的一个shell程序,支持IO重定向,管道等操作
\end{itemize}
%
% % ---

%%%%%%%%%%%%%%%%%%%%%%%%%%%%%%%%%%%%%%%%
%            abandoned                 %
%%%%%%%%%%%%%%%%%%%%%%%%%%%%%%%%%%%%%%%%
% ---
% \datedproject{TCC}{课内项目}{2020年2月 -- 2020年3月}
% {\it C语言编译器前端}
% \quad \href{https://github.com/colinaaa/hello-ds}{{\color{gray}{\faLink}}~}
% \quad \emph{核心开发者}
% \vspace{0.4ex}
% \\
% 数据结构课程设计,实现了一个C子集的编译器前端。
% \begin{itemize}
%   \item 编译原理(前端部分),核心代码部分可以自举
%   \item 词法分析,语法分析(Top down parser, 递归下降),语义分析(类型判断,常量优化)
% \end{itemize}
%
% % ---
% \datedproject{CSAPP Labs}{课内项目}{2020年1月 -- 2020年3月}
% {\it CMU 15213 Introduction to Computer System课程实验}
% % \quad \href{https://github.com/colinaaa/hello-ds}{{\color{gray}{\faLink}}~}
% \quad \emph{核心开发者}
% \vspace{0.4ex}
% \\
% 著名的有关操作系统基础知识的几个实验,全部独立完成。
% \begin{itemize}
%   \item x86-64汇编的阅读与编写,包括逆向和基础的攻击(栈溢出,Return-oriented programming)
%   \item 实现了一个模拟x86-64虚拟内存的虚拟机,并优化代码,以提高其缓存命中率
%   \item 实现了支持单线程的{\tt malloc}函数,使用了多个双向循环内核链表,从而兼顾了分配的速率与碎片产生率
%   \item 实现了Linux下的一个shell程序,支持IO重定向,管道等操作
% \end{itemize}

% ----------------------------------------------------------------------------- %

\section{社团经历}
\dateditem{\textbf{华中科技大学联创团队}}{2018年10月 -- 至今}

\end{document}
