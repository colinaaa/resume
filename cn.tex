%!TEX program = xelatex

\documentclass{uniquecv}

\usepackage{fontawesome5}
\usepackage{fontspec}
\usepackage[colorlinks]{hyperref}
% ----------------------------------------------------------------------------- %

%\setmonofont{FiraCode}
\begin{document}
\name{王清雨}
\medskip

\basicinfo{
  \faPhone ~ (+86) 136-5113-2812
  \textperiodcentered\
  \faEnvelope ~ qingyu.wang@aliyun.com
  \textperiodcentered\
  \faGithub ~ github.com/colinaaa
  \textperiodcentered\
  \faBlog ~ \href{https://about.outsiders.top}{\color{black}{blog}}
}

% ----------------------------------------------------------------------------- %

\section{教育背景}
\dateditem{\textbf{清华大学附属中学} \quad 初中、高中}{2012年 -- 2018年 }
成绩:年级前10\%
\dateditem{\textbf{华中科技大学} \quad 计算机科学与技术\quad 本科}{2018年 -- 至今 }
成绩:年级前20\% \quad 英语:CET4-573分 \quad CET6-551分


% ----------------------------------------------------------------------------- %

\section{专业技能}
\smallskip
\textbf{Programming}
\quad C++1a/Golang/TypeScript/React/Linux/操作系统/数据结构/计算机网络

\textbf{Tools} \quad \LaTeX/(neo)vim/Git

% ----------------------------------------------------------------------------- %

\section{实习经历}
\dateditem{\textbf{北京字节跳动科技有限公司} \quad 抖音直播前端实习生} {2020年 12月 - 2021年 3月}

\begin{itemize}
  \item 独立负责抖音直播端内红包页面开发、重构
  \item 参与跨端框架ReactLynx的建设
\end{itemize}

\dateditem{\textbf{腾讯科技(北京)有限公司} \quad CSIG后端实习生} {2020年 7月 - 2020年 9 月}

\begin{itemize}
  \item 为诺华(Novartis)搭建一个后台Dashboard
  \item 推动组内基础设施方面的建设,包括 CI/CD 等
\end{itemize}

% ----------------------------------------------------------------------------- %

\section{项目经历}

\datedproject{Unique HR系统}{团队项目}{2020年2月 - 至今}
{\it 联创团队招新系统前端与后台}
\quad \emph{主要维护者}
\vspace{0.4ex}

\begin{itemize}
  \item 前端技术栈 React/TypeScript/Redux/RxJS/Material UI
  \item 后端技术栈 Node.js/TypeScript/Mongoose/Express.js
\end{itemize}

\datedproject{xv6 Labs}{课内项目}{2020年9月 -- 2020年11月}
{\it MIT 6.S081 Operating System Engineering 课程实验}
\vspace{0.4ex}
\\
MIT操作系统课程实验,基于UNIX v6 与 RISC-V,全部独立完成。
\begin{itemize}
  \item 实现了一个用户态的page table,使得在内核不需要页表切换即可访问用户态内存。
  \item 实现了lazy allocation,通过page fault来延后分配物理内存的时间。
  \item 实现了copy on write,通过page fault与page table来提升fork的效率。
\end{itemize}

\datedproject{CSAPP Labs}{课内项目}{2020年1月 -- 2020年3月}
{\it CMU 15213 Introduction to Computer System课程实验}
\vspace{0.4ex}
\begin{itemize}
  \item AMD-64 Assembly / Cache / malloc / Shell
\end{itemize}
%
% % ---

%%%%%%%%%%%%%%%%%%%%%%%%%%%%%%%%%%%%%%%%
%            abandoned                 %
%%%%%%%%%%%%%%%%%%%%%%%%%%%%%%%%%%%%%%%%
% ---
% \datedproject{TCC}{课内项目}{2020年2月 -- 2020年3月}
% {\it C语言编译器前端}
% \quad \href{https://github.com/colinaaa/hello-ds}{{\color{gray}{\faLink}}~}
% \quad \emph{核心开发者}
% \vspace{0.4ex}
% \\
% 数据结构课程设计,实现了一个C子集的编译器前端。
% \begin{itemize}
%   \item 编译原理(前端部分),核心代码部分可以自举
%   \item 词法分析,语法分析(Top down parser, 递归下降),语义分析(类型判断,常量优化)
% \end{itemize}
%
% % ---
% \datedproject{CSAPP Labs}{课内项目}{2020年1月 -- 2020年3月}
% {\it CMU 15213 Introduction to Computer System课程实验}
% % \quad \href{https://github.com/colinaaa/hello-ds}{{\color{gray}{\faLink}}~}
% \quad \emph{核心开发者}
% \vspace{0.4ex}
% \\
% 著名的有关操作系统基础知识的几个实验,全部独立完成。
% \begin{itemize}
%   \item x86-64汇编的阅读与编写,包括逆向和基础的攻击(栈溢出,Return-oriented programming)
%   \item 实现了一个模拟x86-64虚拟内存的虚拟机,并优化代码,以提高其缓存命中率
%   \item 实现了支持单线程的{\tt malloc}函数,使用了多个双向循环内核链表,从而兼顾了分配的速率与碎片产生率
%   \item 实现了Linux下的一个shell程序,支持IO重定向,管道等操作
% \end{itemize}

% ----------------------------------------------------------------------------- %

\section{社团经历}
\dateditem{\textbf{华中科技大学联创团队}}{2018年10月 -- 至今}

\end{document}
